\section{Introduction}

Consider two statements about international relations: (1) \textit{``Western diplomats met with Ukrainian officials to discuss potential peace negotiations,''} versus (2) \textit{``While Western officials claim to pursue peace, their negotiations appear primarily designed to serve their own geopolitical interests rather than address regional security concerns.''} While the first presents factual reporting, the second exemplifies a narrative frame: \textit{Discrediting the West: Diplomacy}. This narrative subtly casts doubt on Western intentions through framing choices—questioning motives (``claim to pursue''), suggesting ulterior purposes (``primarily designed to serve their own interests''), and implying disregard for legitimate concerns. Detecting such narratives requires hierarchical classification: identifying the top-level frame (\textit{Discrediting the West}), the specific narrative dimension (\textit{Diplomacy}), and potentially nested sub-narratives (e.g., \textit{questioning motives}, \textit{serving own interests}). Such multi-layered structures are pervasive in online discourse, where persuasive messaging often combines several rhetorical strategies simultaneously to influence public opinion.

Text Classification (TC) is a foundational task in Natural Language Processing (NLP) \citep{Zangari2024}. While traditional approaches often model TC as a single-label problem, real-world texts frequently contain multiple overlapping themes, motivating the use of Multi-Label Classification (MLC) \citep{Hu2025,TidakeSane2018}. A particularly challenging yet crucial variant is Hierarchical Multi-Label Classification (HMLC), where labels are organized in a predefined hierarchy (e.g., a tree or DAG) \citep{liu2023recentadvanceshierarchicalmultilabel}. Narrative detection exemplifies this challenge: narratives are structured frameworks that shape how information is presented and interpreted, often appearing in nested relationships where broad themes (e.g., ``climate skepticism'') contain more specific sub-narratives (e.g., ``scientists are biased,'' ``data is manipulated''). Accurately identifying these nested structures is essential for understanding persuasive communication in misinformation, news framing, and political discourse.

The advent of Large Language Models (LLMs) has opened new frontiers for HMLC, enabling powerful zero-shot classification without extensive labeled data. However, this power comes with significant reliability challenges. LLMs are known to be stochastic, producing different outputs for the same input, and often exhibit low instruction fidelity, failing to consistently adhere to complex hierarchical constraints or output formats \citep{Qin2024InFoBench}. These limitations hinder their deployment in high-stakes applications where robustness and verifiable reasoning are paramount.

To address these shortcomings, we introduce Agora, a multi-agent ensemble framework that significantly improves the robustness and accuracy of Large Language Models on complex Hierarchical Multi-Label Classification tasks. By aggregating parallel classifications from multiple LLM agents via voting, Agora mitigates the inherent stochasticity of LLMs and produces more reliable results than a standard single-agent approach. We evaluate our framework on SemEval-2025 Task 10 for multilingual narrative classification \citep{semeval2025task10}, demonstrating state-of-the-art performance across five languages.

Our main contributions are:
\begin{itemize}
\item A multi-agent ensemble framework (Agora) that uses consensus-based voting to improve LLM reliability for hierarchical classification, achieving first place in Hindi and top-3 rankings in four other languages on SemEval-2025 Task 10.
\item A comprehensive comparison demonstrating that consensus-based ensembles outperform both naive baselines and actor-critic self-refinement approaches, with an 11\% F1-Samples improvement over single-agent classification.
\item A systematic investigation of voting strategies (union, intersection, majority vote) showing that intersection voting provides the best precision-recall balance for hierarchical narrative detection.
\item Fully reproducible prompt templates and implementation details provided in the appendix to facilitate adoption and extension of the approach.
\end{itemize}