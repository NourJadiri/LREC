\section{Limitations and Future Work}

\subsection{Limitations}

Our study, while demonstrating the effectiveness of the Agora framework, is subject to several limitations that offer avenues for future work.

\textbf{Model and API Dependency:} Our zero-shot framework relies on proprietary LLMs (e.g., gpt-5-nano). This limits full reproducibility, as performance is tied to specific, closed-source model versions. Future work should explore the effectiveness of this ensemble approach with open-source models to ensure broader accessibility and control.

\textbf{Prompt Sensitivity:} The performance of any zero-shot system is highly sensitive to prompt engineering. While we standardized prompts across configurations, it is possible that the specific phrasing influenced the degree of stochasticity observed and the effectiveness of the voting mechanism.

\textbf{Computational Cost:} Deploying a multi-agent ensemble incurs a direct multiplication of inference cost and latency compared to a single-agent system. In our $N=3$ configuration, the cost is roughly triple that of the baseline. While the performance gains justified this trade-off in a competitive setting, a critical area for future research is exploring cost-reduction techniques, such as using smaller, distilled models for some agents or implementing more sophisticated routing where an ensemble is only triggered for high-uncertainty cases.

\subsection{Future Work}

Based on our findings, we propose several directions for future research. First, exploring a weighted majority vote, where agents' votes are weighted by a confidence score, could further refine the consensus mechanism. Second, a hybrid approach that uses the fast Naive Baseline for simple cases and dynamically invokes the more robust Agora ensemble for difficult or ambiguous texts could optimize the cost-performance trade-off. Finally, applying the Agora framework to other complex, hierarchical domains beyond propaganda detection, such as legal text or biomedical literature analysis, would further validate its generalizability.

Beyond the zero-shot paradigm, a promising direction involves fine-tuning LLMs on task-specific data to achieve higher performance. However, this approach faces a significant challenge in the multilingual propaganda detection context: the need for high-quality labeled data across multiple languages while preserving cultural nuances. As noted in \citep{eljadiri-nurbakova-2025-team}'s paper, machine translation of training data often loses cultural context and domain-specific subtleties, making direct translation-based data augmentation problematic. Future work must explore methods for generating culturally-aware training data, such as collaborative annotation with native speakers or culturally-grounded synthetic data generation, to enable effective fine-tuning while respecting the nuanced differences between languages.