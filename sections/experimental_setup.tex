\section{Experimental setup}

\subsection{Dataset and its Challenges}

The dataset from the SemEval-2025 Task 10, Subtask 2, a task focused on multilingual propaganda narrative detection \cite{semeval2025task10}, spans five languages. While we focus our evaluation on the English subset, our analysis of this task reveals several key characteristics that motivate our architectural choices.

First, the dataset is fundamentally multi-label in nature. A majority of documents (54.0\%) are assigned more than one narrative, with an average of 2.28 labels per document. This necessitates a multi-label modeling approach.

Second, and most critically, the dataset exhibits a severe class imbalance. As illustrated in Appendix Figure~\ref{fig:narrative_distribution}, the 22 narrative labels follow a classic long-tailed distribution. The most frequent narrative appears 65 times more often than the least frequent one. This sparsity poses a significant challenge for traditional supervised models, which risk overfitting on the few ``head'' classes and failing to generalize to the many rare but meaningful ``tail'' classes. This characteristic strongly motivates our adoption of a zero-shot paradigm, which does not depend on label frequencies in a training set.

Due to the multi-label nature and severe class imbalance inherent in the dataset, we decided to adopt a zero-shot learning approach. This paradigm avoids dependency on large per-class training counts and is well-suited to handle the long-tail distribution of narratives, making it ideal for this challenging classification task.

\subsection{Dataset Statistics}

The SemEval-2025 Task 10, Subtask 2 dataset comprises 1,699 documents spanning five languages: Bulgarian (BG), English (EN), Hindi (HI), Portuguese (PT), and Russian (RU). The dataset contains a total of 3,874 narrative annotations and 3,874 corresponding subnarrative annotations, organized in a two-level hierarchy with 22 top-level narratives and 95 subnarratives.

\textbf{Multi-label Complexity.} The dataset exhibits significant multi-label characteristics: 918 documents (54.0\%) are assigned multiple narratives, while 781 documents (46.0\%) have a single narrative label. On average, each document is annotated with 2.28 narratives and 2.28 subnarratives. The maximum complexity reaches 14 narratives in a single document, highlighting the challenge of capturing diverse narrative content.

\textbf{Class Imbalance.} A severe class imbalance characterizes the dataset, with the most frequent narrative (\textit{URW: Discrediting Ukraine}) appearing 584 times, while the least frequent (\textit{CC: Green policies are geopolitical instruments}) appears only 9 times. This yields an imbalance ratio of 64.9:1, reinforcing the motivation for zero-shot approaches that do not depend on per-class training counts.

\textbf{Cross-lingual Distribution.} The language-specific statistics reveal variation in annotation density: Bulgarian (401 docs, 855 annotations, avg 2.13 labels/doc), English (399 docs, 875 annotations, avg 2.19 labels/doc), Hindi (366 docs, 655 annotations, avg 1.79 labels/doc), Portuguese (400 docs, 1,217 annotations, avg 3.04 labels/doc), and Russian (133 docs, 272 annotations, avg 2.05 labels/doc). Portuguese exhibits notably higher annotation density, while Hindi shows lower average labels per document.

\textbf{Text Characteristics.} Document lengths vary considerably, ranging from 45 to 4,422 words, with an average of 404 words and a median of 371 words. This variation requires robust handling of both short and long-form content.

\subsection{Hardware Configuration}

All experiments were conducted on a machine with the following specifications: {Processor:} Intel Core 9 Ultra; \textbf{GPU:} NVIDIA RTX 4070 (8GB VRAM); \textbf{Memory:} 32GB RAM; \textbf{Storage:} 1TB SSD.

\subsection{Evaluation Scope}

We primarily evaluate the performance of our models on the English test set, providing detailed analysis and interpretation of the results on this language. However, to assess the multilingual capabilities of our approaches, we also conducted experiments on additional languages to validate the generalizability of our methods across different linguistic contexts.