% LREC 2026 Paper
% LLM Pipeline for Hierarchical Narrative Classification
\documentclass[10pt, a4paper]{article}

\usepackage[review]{lrec2026} % Use 'review' for submission, 'final' for camera-ready

\title{LLM Pipeline for Hierarchical Narrative Classification: A Traceable Approach to Multilingual Propaganda Detection}

\name{Author1, Author2, Author3} 

\address{Affiliation1, Affiliation2, Affiliation3 \\
         Address1, Address2, Address3 \\
         author1@xxx.yy, author2@zzz.edu, author3@hhh.com\\
         \{author1, author5, author9\}@abc.org\\}

\abstract{
This paper presents Agora, a configurable and reproducible framework for robust hierarchical multi-label classification using large language models. We address critical reliability challenges in LLM-based classification through a multi-agent ensemble approach with voting consensus and an optional actor-critic self-refinement loop. Our system decomposes the hierarchical classification task into manageable steps via a state-graph pipeline, providing enhanced traceability for multilingual propaganda detection. We evaluate our approach on hierarchical narrative classification benchmarks, demonstrating improved robustness and consistency compared to baseline LLM approaches.
 \\ \newline \Keywords{hierarchical classification, propaganda detection, narrative analysis, large language models, multilingual NLP} }

\begin{document}

\maketitleabstract

\section{Introduction}

Text Classification (TC) is a foundational task in Natural Language Processing (NLP) \citep{Zangari2024}. While traditional approaches often model TC as a single-label problem, real-world texts frequently contain multiple overlapping themes, motivating the use of Multi-Label Classification (MLC) \citep{Hu2025,TidakeSane2018}. A particularly challenging yet crucial variant is Hierarchical Multi-Label Classification (HMLC), where labels are organized in a predefined hierarchy (e.g., a tree or DAG) \citep{liu2023recentadvanceshierarchicalmultilabel}. This structure is common in domains requiring nuanced analysis, such as misinformation detection, where identifying nested propaganda narratives is a key challenge.

The advent of Large Language Models (LLMs) has opened new frontiers for HMLC, enabling powerful zero-shot classification without extensive labeled data. However, this power comes with significant reliability challenges. LLMs are known to be stochastic, producing different outputs for the same input, and often exhibit low instruction fidelity, failing to consistently adhere to complex hierarchical constraints or output formats \citep{some_citation_on_llm_brittleness}. These limitations hinder their deployment in high-stakes applications where robustness and verifiable reasoning are paramount.

To address these shortcomings, we introduce Agora, a configurable and reproducible framework for robust hierarchical classification. Our system is designed as a state-graph pipeline that decomposes the HMLC task into a series of manageable steps. Its core novelty lies in two mechanisms designed to enhance LLM reliability: (1) a multi-agent ensemble method where multiple independent LLM agents classify the text in parallel, with their outputs aggregated via a voting mechanism to form a robust consensus; and (2) an optional actor-critic self-refinement loop, where a dedicated ``critic'' agent validates the classifier's outputs and provides feedback for iterative correction.

\section{Related Work}

The application of Large Language Models (LLMs) to Hierarchical Multi-Label Classification (HMLC) has rapidly matured, moving beyond complex supervised architectures that explicitly encode label hierarchies with Graph Neural Networks \citep{zhou-etal-2020-hierarchy} towards more flexible zero-shot paradigms \citep{wang-etal-2023-text2topic}. The recent SemEval-2025 Task 10 on multilingual narrative detection serves as a clear benchmark for the current state-of-the-art. Top-performing systems demonstrate the viability of zero-shot LLM approaches, utilizing techniques such as specialized prompting strategies and retrieval-augmented generation to handle the two-level taxonomy \citep{singh-etal-2025-gatenlp,younus-qureshi-2025-nlptuducd}.

In our own prior work on this task, we introduced a modular ``agentic'' framework that decomposed the HMLC problem into a set of parallel binary decisions, with specialized LLM agents assigned to individual labels \citep{eljadiri-nurbakova-2025-team}. While this and other systems have proven effective, they predominantly rely on the output of a single LLM agent or a single generative pass for their final predictions. This exposes them to the inherent stochasticity of LLMs, where identical inputs can yield different classifications across runs, posing a significant challenge for reliability. The use of ensembles to improve robustness is a well-established technique in machine learning \citep{Read2021} and has been effective for fine-tuned Transformer models in similar propaganda detection tasks \citep{jurkiewicz-etal-2020-applicaai}. However, the systematic application of ensembling to mitigate the unreliability of modern, zero-shot LLM systems in a complex HMLC context remains underexplored. Our work directly addresses this gap, proposing a multi-agent ensemble framework that aggregates votes to produce a more stable, consensus-driven classification.


% Add more sections as needed
% \input{sections/methodology.tex}
% \input{sections/experiments.tex}
% \section{Results}

In this section, we present the empirical results of our experiments. We first provide a detailed architectural comparison to understand the source of improvements, analyze voting strategies, and then validate our approach through competitive performance in the official SemEval-2025 Task 10 shared task.

\subsection{Architectural Comparison and Ablation Study}

To understand the effectiveness of our approach, we conducted a controlled comparison on the English dataset between our final Agora system, an intermediate Actor-Critic pipeline, and a Naive Baseline. The results are presented in Table~\ref{tab:main_comparison}.

\begin{table*}[ht]
\centering
\caption{Main Performance Comparison on the English Dataset (F1-Samples).}
\label{tab:main_comparison}
\begin{tabular}{lcc}
\hline
\textbf{System Configuration} & \textbf{F1-Samples Score} & \textbf{Improvement over Baseline} \\
\hline
1. Naive Baseline (Single-Pass) & 0.382 & --- \\
2. Actor-Critic Pipeline & 0.417 & +0.035 \\
3. Agora (3-Agent Intersection) & \textbf{0.424} & \textbf{+0.042} \\
\hline
\end{tabular}
\end{table*}

This ablation study reveals a clear performance pathway. The Naive Baseline represents the H3Prompting approach described in Section~\ref{sec:baseline}: a single agent per level, applying one LLM call per hierarchical step without refinement. This baseline achieves an F1-Samples score of 0.382.

The Actor-Critic Pipeline provides a significant improvement of +0.035, demonstrating that a self-refinement loop can effectively correct errors and improve classification quality.

However, our proposed Agora framework delivers an even greater performance gain. It achieves the highest F1-Samples score of 0.424, a +0.042 point improvement (+11.0\%) over the Naive Baseline. This result confirms our central hypothesis: while single-agent refinement is beneficial, the consensus-based approach of a multi-agent ensemble is a more robust and superior method for enhancing LLM reliability.

\subsection{Analysis of Agora's Voting Strategies}

To isolate the effect of the ensemble itself, we compared a single-agent Agora configuration (``Vanilla'') against the different multi-agent voting strategies. Table~\ref{tab:voting_strategies} shows that the choice of aggregation method is critical to the ensemble's success.

\begin{table*}[!ht]
\centering
\caption{Ablation of Agora Configurations on the English Dataset (F1-Samples).}
\label{tab:voting_strategies}
\begin{tabular}{lcc}
\hline
\textbf{Agora Configuration} & \textbf{F1-Samples Score} & \textbf{Improvement over Single Agent} \\
\hline
1. Single Agent (Vanilla) & 0.389 & --- \\
2. 3-Agent Union & 0.375 & -0.014 \\
3. 3-Agent Majority Vote & 0.402 & +0.013 \\
4. 3-Agent Intersection & \textbf{0.424} & \textbf{+0.035} \\
\hline
\end{tabular}
\end{table*}

This analysis provides a crucial insight: simply combining agent outputs is not enough. The Union strategy, which naively accepts all proposed labels, actually performs worse than a single agent by aggregating noise. In contrast, the consensus-based mechanisms deliver clear benefits. Both Majority Vote (+0.013) and Intersection (+0.035) significantly outperform the single agent. The Intersection strategy, requiring unanimous agreement, proves most effective for this task, successfully filtering out stochastic, low-confidence predictions to achieve the best overall performance. This demonstrates that the core advantage of Agora lies in its ability to systematically reduce noise through consensus.

\subsection{State-of-the-Art Performance in SemEval-2025}

The ultimate validation of our approaches comes from their performance in the official SemEval-2025 Task 10 competition. We submitted two architectures: the Actor-Critic pipeline (Gemini 2.5 Flash) and the Agora multi-agent ensemble (GPT-5-nano). Table~\ref{tab:semeval_comprehensive} presents a comprehensive comparison across all five languages.

\begin{table*}[ht]
\centering
\caption{Comprehensive comparison of Actor-Critic and Agora architectures in SemEval-2025 Task 10 across all languages. F1 Samples is the primary evaluation metric, with F1 Macro Coarse provided for reference. Rankings reflect official competition standings. Agora configurations shown represent the best-performing variant for each language.}
\label{tab:semeval_comprehensive}
\small
\begin{tabular}{lcccccccc}
\hline
\textbf{Language} & \multicolumn{3}{c}{\textbf{Actor-Critic (Gemini 2.5 Flash)}} & \multicolumn{4}{c}{\textbf{Agora (GPT-5-nano)}} & \textbf{$\Delta$ F1} \\
\cline{2-4} \cline{5-8}
& \textbf{F1 Samples} & \textbf{F1 Macro} & \textbf{Rank} & \textbf{F1 Samples} & \textbf{F1 Macro} & \textbf{Config} & \textbf{Rank} & \textbf{Samples} \\
\hline
Bulgarian & 0.381 & 0.590 & 5th & 0.403 & 0.575 & Narr. Inter. & 4th & +0.022 \\
English & 0.433 & 0.547 & 2nd & 0.424 & 0.518 & Full Inter. & 2nd & -0.009 \\
Hindi & 0.435 & 0.515 & 3rd & \textbf{0.581} & \textbf{0.673} & Full Inter. & \textbf{1st} & \textbf{+0.146} \\
Portuguese & 0.433 & 0.679 & 2nd & 0.385 & 0.602 & Narr. Union & 2nd & -0.048 \\
Russian & 0.410 & 0.589 & 4th & 0.437 & 0.556 & Narr. Union & 3rd & +0.027 \\
\hline
\textbf{Average} & \textbf{0.418} & \textbf{0.584} & \textbf{3.2} & \textbf{0.446} & \textbf{0.585} & --- & \textbf{2.4} & \textbf{+0.028} \\
\hline
\end{tabular}
\end{table*}

The comprehensive comparison reveals several critical insights about the two architectures. On average, Agora achieves a +0.028 improvement in F1 Samples over the Actor-Critic pipeline (0.446 vs 0.418), while maintaining comparable F1 Macro performance (0.585 vs 0.584). More importantly, Agora's improved average ranking from 3.2 to 2.4 demonstrates superior competitive performance.

\textbf{Hindi shows dramatic improvement.} The most striking result is Hindi, where Agora's Full Intersection configuration achieves 0.581 F1 Samples (+0.146 improvement, +33.6\%) and 0.673 F1 Macro, elevating the ranking from 3rd to 1st place. This substantial gain suggests that consensus-based ensembling is particularly effective for languages where single-agent predictions exhibit higher variance.

\textbf{Configuration selection is language-dependent.} The optimal Agora configuration varies significantly across languages: Bulgarian benefits from Narrative Intersection, English and Hindi achieve best results with Full Intersection, while Portuguese and Russian perform optimally with Narrative Union. This pattern indicates that different languages and their associated narrative distributions require different consensus strategies, with Full Intersection particularly effective for languages like Hindi with high prediction variance.

\textbf{Mixed performance on English and Portuguese.} Interestingly, Agora shows slight degradation on English (-0.009) and more substantial decrease on Portuguese (-0.048) despite maintaining competitive rankings. This suggests that for languages with certain characteristics, the Actor-Critic's iterative refinement approach may be more effective than consensus-based voting, particularly when the base model predictions are already highly accurate.

\textbf{Consistent ranking improvements validate ensemble approach.} Despite mixed F1 Samples results, Agora achieves better or equal rankings in all languages, most notably improving Bulgarian (5th→4th) and Russian (4th→3rd) beyond the dramatic Hindi improvement (3rd→1st). This validates that consensus-based ensembling provides more robust competitive performance across diverse multilingual settings.

These results confirm that both approaches represent state-of-the-art methods for multilingual hierarchical narrative classification, with Agora's consensus mechanism achieving superior average performance (+0.028 F1 Samples) and competitive standings (2.4 vs 3.2 average rank), particularly excelling in languages with high prediction variance like Hindi.


% \section{Conclusion}

We addressed the challenge of stochasticity in Large Language Models applied to Hierarchical Multi-Label Classification by introducing Agora, a multi-agent ensemble framework that leverages consensus-based voting to transform unreliable single-agent classification into a robust decision process.

Our experiments on SemEval-2025 Task 10 revealed three key findings: (1) naive single-pass LLM baselines are outperformed by sophisticated architectures, (2) Actor-Critic self-refinement can introduce noise and degrade performance due to critic unreliability, and (3) Agora with 3-agent voting delivers substantial performance gains across all languages, achieving first-place rankings in Hindi and competitive top-tier performance overall.

These results validate our central claim: multi-agent ensembling through consensus is a practical and effective method for building robust zero-shot classification systems. As NLP increasingly relies on powerful but imperfect LLMs, frameworks prioritizing reliability through consensus will be essential for trustworthy and deployable solutions.


\section*{Acknowledgements}

Place all acknowledgments here.

\section*{References}

\bibliographystyle{lrec2026-natbib}
\bibliography{references,acl_anthology}

\end{document}

%%% Local Variables:
%%% mode: latex
%%% TeX-master: t
%%% End:
