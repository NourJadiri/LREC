% LREC 2026 Paper
% LLM Pipeline for Hierarchical Narrative Classification
\documentclass[10pt, a4paper]{article}

\usepackage[review]{lrec2026} % Use 'review' for submission, 'final' for camera-ready

\title{LLM Pipeline for Hierarchical Narrative Classification: A Traceable Approach to Multilingual Propaganda Detection}

\name{Author1, Author2, Author3} 

\address{Affiliation1, Affiliation2, Affiliation3 \\
         Address1, Address2, Address3 \\
         author1@xxx.yy, author2@zzz.edu, author3@hhh.com\\
         \{author1, author5, author9\}@abc.org\\}

\abstract{
This paper presents Agora, a configurable and reproducible framework for robust hierarchical multi-label classification using large language models. We address critical reliability challenges in LLM-based classification through a multi-agent ensemble approach with voting consensus and an optional actor-critic self-refinement loop. Our system decomposes the hierarchical classification task into manageable steps via a state-graph pipeline, providing enhanced traceability for multilingual propaganda detection. We evaluate our approach on hierarchical narrative classification benchmarks, demonstrating improved robustness and consistency compared to baseline LLM approaches.
 \\ \newline \Keywords{hierarchical classification, propaganda detection, narrative analysis, large language models, multilingual NLP} }

\begin{document}

\maketitleabstract

\section{Introduction}

Text Classification (TC) is a foundational task in Natural Language Processing (NLP) \citep{Zangari2024}. While traditional approaches often model TC as a single-label problem, real-world texts frequently contain multiple overlapping themes, motivating the use of Multi-Label Classification (MLC) \citep{Hu2025,TidakeSane2018}. A particularly challenging yet crucial variant is Hierarchical Multi-Label Classification (HMLC), where labels are organized in a predefined hierarchy (e.g., a tree or DAG) \citep{liu2023recentadvanceshierarchicalmultilabel}. This structure is common in domains requiring nuanced analysis, such as misinformation detection, where identifying nested propaganda narratives is a key challenge.

The advent of Large Language Models (LLMs) has opened new frontiers for HMLC, enabling powerful zero-shot classification without extensive labeled data. However, this power comes with significant reliability challenges. LLMs are known to be stochastic, producing different outputs for the same input, and often exhibit low instruction fidelity, failing to consistently adhere to complex hierarchical constraints or output formats \citep{some_citation_on_llm_brittleness}. These limitations hinder their deployment in high-stakes applications where robustness and verifiable reasoning are paramount.

To address these shortcomings, we introduce Agora, a configurable and reproducible framework for robust hierarchical classification. Our system is designed as a state-graph pipeline that decomposes the HMLC task into a series of manageable steps. Its core novelty lies in two mechanisms designed to enhance LLM reliability: (1) a multi-agent ensemble method where multiple independent LLM agents classify the text in parallel, with their outputs aggregated via a voting mechanism to form a robust consensus; and (2) an optional actor-critic self-refinement loop, where a dedicated ``critic'' agent validates the classifier's outputs and provides feedback for iterative correction.

\section{Related Work}

The core challenge in Multi-Label Classification (MLC) has historically been the effective modeling of inter-label dependencies. Early problem-transformation methods sought to adapt single-label algorithms to this task. Approaches ranged from Binary Relevance, which simplifies the problem by training an independent classifier for each label but ignores correlations, to Classifier Chains, which attempt to capture dependencies by feeding the predictions of one classifier as features to the next in a sequence \citep{zhang_binary_2018,Read2011}. While foundational, these methods were often superseded by deep learning models, particularly Transformers, which could learn complex label correlations implicitly from large datasets \citep{devlin_bert_2019}.

Building on these advances, research into Hierarchical Multi-Label Classification (HMLC) focused on creating architectures that explicitly leverage the label taxonomy. A dominant paradigm involved coupling a Transformer-based text encoder with a Graph Neural Network (GNN) that operates on the label graph, allowing the model to learn hierarchy-aware representations and enforce structural consistency \citep{zhou-etal-2020-hierarchy,xu-etal-2021-hierarchical}. However, the advent of Large Language Models (LLMs) has fundamentally shifted the landscape, enabling powerful zero-shot HMLC capabilities that bypass the need for complex, task-specific architectures and extensive supervised training \citep{wang-etal-2023-text2topic}.

The recent SemEval-2025 Task 10 on narrative detection \citep{semeval2025task10} exemplifies this modern, LLM-driven approach. State-of-the-art systems effectively use techniques like retrieval-augmented generation and specialized prompting to classify texts within the task's two-level hierarchy \citep{singh-etal-2025-gatenlp,younus-qureshi-2025-nlptuducd}. Prior work has also introduced modular ``agentic'' frameworks that decompose the task into parallel binary decisions \citep{eljadiri-nurbakova-2025-team}. Yet, these pioneering systems share a fundamental vulnerability: their reliance on a single, non-deterministic generative pass, which exposes them to the inherent stochasticity of LLMs. 

While ensembling is a well-established technique for improving robustness in supervised models \citep{jurkiewicz-etal-2020-applicaai}, its systematic application to mitigate the unreliability of zero-shot LLMs in HMLC remains underexplored. This represents a critical research gap: the stochasticity of individual LLM calls stands in stark contrast to the deterministic requirements of hierarchical classification systems, yet few works have investigated consensus-driven approaches to address this mismatch. Our work directly addresses this gap, proposing a multi-agent ensemble framework to produce stable, consensus-driven classifications that can reliably handle the demanding constraints of hierarchical multilabel detection.


% Add more sections as needed
% \input{sections/methodology.tex}
% \input{sections/experiments.tex}
% \section{Results}

In this section, we present the empirical results of our experiments. We first demonstrate the competitive performance of the Agora framework through official SemEval-2025 rankings. We then provide a detailed architectural comparison to understand the source of Agora's success, and conclude with an analysis of its voting strategies.

\subsection{State-of-the-Art Performance in SemEval-2025}

The ultimate validation of the Agora framework comes from its performance in the official SemEval-2025 Task 10 competition. Our submitted system, which was the Agora multi-agent ensemble, achieved top-tier rankings across all five languages, including a first-place finish in Hindi.

\begin{table}[ht]
\centering
\caption{Official INSALyon2 Rankings using the Agora Framework in SemEval-2025 Task 10.}
\label{tab:semeval_rankings}
\begin{tabular}{lc}
\hline
\textbf{Language} & \textbf{Rank} \\
\hline
Hindi & \textbf{1st} \\
English & 2nd \\
Portuguese & 2nd \\
Russian & 3rd \\
Bulgarian & 4th \\
\hline
\end{tabular}
\end{table}

These results confirm that the consensus-based approach is not just a theoretical improvement but a practical, state-of-the-art method for complex HMLC tasks. Achieving a first-place and two second-place finishes in a highly competitive shared task validates the robustness and generalizability of the Agora architecture.

Notably, these rankings represent a substantial improvement over our earlier submission using the Actor-Critic pipeline approach. Table~\ref{tab:actor_critic_rankings} shows the performance of that intermediate architecture.

\begin{table}[ht]
\centering
\caption{Official INSALyon2 Rankings using the Actor-Critic Pipeline in SemEval-2025 Task 10.}
\label{tab:actor_critic_rankings}
\begin{tabular}{lc}
\hline
\textbf{Language} & \textbf{Rank} \\
\hline
English & 2nd \\
Portuguese & 2nd \\
Hindi & 3rd \\
Russian & 4th \\
Bulgarian & 5th \\
\hline
\end{tabular}
\end{table}

While the Actor-Critic approach already achieved competitive results with consistent top-5 rankings, the transition to Agora's multi-agent ensemble brought further improvements, most notably elevating Hindi from 3rd to 1st place. This progression demonstrates the value of moving from a single-critic refinement paradigm to a consensus-based ensemble approach.

\subsection{Architectural Comparison and Ablation Study}

To understand the source of Agora's success, we conducted a controlled comparison on the English dataset between our final Agora system, an intermediate Actor-Critic pipeline, and a Naive Baseline.

\begin{table*}[ht]
\centering
\caption{Main Performance Comparison on the English Dataset. F1-Scores are reported for both hierarchy levels.}
\label{tab:main_comparison}
\begin{tabular}{lcc}
\hline
\textbf{System Configuration} & \textbf{Narrative F1} & \textbf{Sub-narrative F1} \\
\hline
1. Naive Baseline (Single-Pass) & 0.513 & 0.382 \\
2. Actor-Critic Pipeline & 0.524 (+0.011) & 0.368 (-0.014) \\
3. Agora (3-Agent Intersection) & 0.518 (+0.005) & \textbf{0.424 (+0.042)} \\
\hline
\end{tabular}
\end{table*}

This ablation study clearly reveals the performance pathway. The Naive Baseline establishes the performance of a standard zero-shot approach. The Actor-Critic Pipeline provides an inconsistent improvement, slightly boosting narrative F1 but harming sub-narrative F1, confirming that a single, fallible critic can introduce noise and is not a reliable path to improvement.

The Agora framework delivers a decisive and consistent performance gain, particularly at the more challenging sub-narrative level. It achieves the highest sub-narrative F1-score of 0.424, an 11.0\% relative improvement over the Naive Baseline. This confirms that aggregating the consensus of multiple parallel agents is a more robust and effective method for enhancing LLM reliability than a single self-correction loop. The performance lift observed here provides a clear explanation for Agora's success in the SemEval competition.

\subsection{Analysis of Voting Strategies}

Within the Agora framework, the choice of voting strategy allows for tuning the system's precision/recall balance. Table~\ref{tab:voting_strategies} shows the performance of different aggregation methods on the English narrative classification task.

\begin{table}[!ht]
\centering
\caption{Ablation of Agora Voting Strategies (English, Narrative F1).}
\label{tab:voting_strategies}
\begin{tabular}{lc}
\hline
\textbf{Aggregation Strategy} & \textbf{F1-Score} \\
\hline
Union & 0.494 \\
Majority Vote (3 Agents) & 0.516 \\
Intersection (3 Agents) & \textbf{0.518} \\
\hline
\end{tabular}
\end{table}

The consensus-based mechanisms (Majority Vote and Intersection) clearly outperform the noisy Union strategy. The Intersection method, requiring unanimous agreement, yields the highest F1-score, indicating its effectiveness at filtering out stochastic, low-confidence predictions from individual agents. This ability to systematically reduce noise through consensus is the core mechanical advantage of the Agora architecture.


% \section{Conclusion}

In this work, we addressed the critical challenge of stochasticity and unreliability in Large Language Models when applied to complex Hierarchical Multi-Label Classification tasks. While the zero-shot capabilities of LLMs are powerful, their inconsistent outputs present a significant barrier to their deployment in real-world, high-stakes applications like propaganda detection.

We introduced Agora, a multi-agent ensemble framework designed to directly mitigate this issue. By leveraging the consensus of multiple independent LLM agents through a robust voting mechanism, Agora transforms a brittle, single-pass classification into a stable, statistically-grounded decision process.

Our experiments, conducted on the challenging SemEval-2025 narrative detection task, provided three key findings. First, we demonstrated that a naive single-pass LLM baseline is outperformed by more sophisticated architectures. Second, we showed that a seemingly intuitive Actor-Critic self-refinement pipeline can be counterproductive, as the critic's own unreliability can introduce noise and degrade performance. Finally, we proved that our Agora framework, using a 3-agent majority vote, delivers substantial and consistent performance gains over both other approaches.

The state-of-the-art performance of Agora in the SemEval test set, validates our central claim: ensembling is a powerful and practical method for building more robust and accurate zero-shot classification systems. As the field increasingly relies on powerful but imperfect LLMs, frameworks like Agora that prioritize reliability through consensus will be essential for creating trustworthy and deployable NLP solutions.

\section*{Acknowledgements}

Place all acknowledgments here.

\section*{References}

\bibliographystyle{lrec2026-natbib}
\bibliography{references,acl_anthology}

\end{document}

%%% Local Variables:
%%% mode: latex
%%% TeX-master: t
%%% End:
